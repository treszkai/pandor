% ========================

\usepackage{newunicodechar}

\newunicodechar{α}{\alpha}
\newunicodechar{γ}{\gamma}
\newunicodechar{δ}{\delta}
\newunicodechar{ε}{\varepsilon}
\newunicodechar{λ}{\lambda}
\newunicodechar{ω}{\omega}

\newunicodechar{Δ}{\Delta}
\newunicodechar{Π}{\Pi}
\newunicodechar{Ω}{\Omega}

\newunicodechar{→}{\rightarrow}
\newunicodechar{⇒}{\Rightarrow}
\newunicodechar{⟨}{\langle}
\newunicodechar{⟩}{\rangle}
\newunicodechar{≤}{\le}
\newunicodechar{≥}{\ge}

\newunicodechar{∈}{\in}

% =========================

\newcommand{\TER}{\mathbf{TER}}
\newcommand{\PC}{\mathbf{PC}}
\newcommand{\LPC}{\mathbf{LPC}}
% \newcommand{\LPCalt}{\mathbf{LG}}
\newcommand{\LTER}{\mathbf{LTER}}
\newcommand{\LTERPC}{\mathbf{LGT}}
\newcommand{\LGT}{\mathbf{LGT}}
\newcommand{\BND}{\mathbf{BND}}
\newcommand{\ACYC}{\mathbf{ACYC}}
\newcommand{\ONE}{\mathbf{ONE}}

\newcommand{\LPCupper}{\overline{{\mathit{LPC}}}}
\newcommand{\LPClower}{\underline{{\mathit{LPC}}}}
\newcommand{\LPCdesired}{{\mathit{LPC}}^{\star}}
\newcommand{\LTERupper}{\overline{{\mathit{LTER}}}}
\newcommand{\LTERlower}{\underline{{\mathit{LTER}}}}
\newcommand{\LTERdesired}{{\mathit{LTER}}^{\star}}
\newcommand{\LTERPCdesired}{{\mathit{LGT}}^{\star}}
\newcommand{\LTERPCupper}{\overline{{\mathit{LGT}}}}
\newcommand{\LTERPClower}{\underline{{\mathit{LGT}}}}
\newcommand{\LGTdesired}{{\mathit{LGT}}^{\star}}
\newcommand{\LGTupper}{\overline{{\mathit{LGT}}}}
\newcommand{\LGTlower}{\underline{{\mathit{LGT}}}}

\newcommand{\Graph}{{G}}

\newcommand{\Env}{\mathcal{E}}
\newcommand{\States}{\mathcal{S}}
\newcommand{\Stts}{\mathcal{S}}
\newcommand{\Acts}{\mathcal{A}}
\newcommand{\Actions}{\mathcal{A}}
\newcommand{\Obss}{\mathcal{O}}
\newcommand{\Observations}{\mathcal{O}}
\newcommand{\Goals}{\mathcal{G}}
\newcommand{\Problem}{\mathcal{P}}
\newcommand{\Prob}{\mathcal{P}}
\newcommand{\GenProb}{\overline{\mathcal{P}}}
\newcommand{\Cont}{\mathcal{C}}

\newcommand{\Xgood}{{\mathrm{good}}}
\newcommand{\Xloop}{{\mathrm{loop}}}
\newcommand{\Xgoal}{{\mathrm{goal}}}
\newcommand{\Xcurrent}{{\mathrm{curr}}}
\newcommand{\Xcurr}{{\mathrm{curr}}}
\newcommand{\Xfail}{{\mathrm{fail}}}
\newcommand{\Xnoter}{{\mathrm{noter}}}

\newcommand{\Xand}{{\mathsf{and}}}
\newcommand{\Xor}{{\mathsf{or}}}
\newcommand{\Xtrue}{{\mathsf{true}}}
\newcommand{\Xfalse}{{\mathsf{false}}}
\newcommand{\Xstop}{\text{\texttt{stop}}}
\newcommand{\Xright}{\text{\texttt{left}}}
\newcommand{\Xleft}{\text{\texttt{right}}}
\newcommand{\Xswin}{s_{\mathrm{win}}}
\newcommand{\Xsfail}{s_{\mathrm{fail}}}
\newcommand{\XSP}{\mathit{SP}}

\newcommand{\ie}{i.e., }
\newcommand{\eg}{e.g., }

\newcommand{\Pandor}{\textsc{Pandor}\ }

\usepackage{amsthm}

% \newtheorem{lemma}[theorem]{Lemma}  % follow theorem's numbering
\newtheorem{lemma}{Lemma}
\newtheorem{theorem}{Theorem}
\newtheorem*{conjecture}{Conjecture}
\newtheorem{corollary}{Corollary}

\theoremstyle{definition}  % bold heading, normal body
\newtheorem{definition}{Definition}
\newtheorem{problem}{Problem}
% \newtheorem{proposition}{Proposition}

\theoremstyle{definition}  % bold heading, normal body, no numbering
\newtheorem*{notation}{Notation}

\theoremstyle{remark}  % italic heading, normal body, no numbering
\newtheorem*{remark}{Remark}


\newcommand{\Xsensed}{\mathit{sensed}}
\newcommand{\Xend}{\mathit{end}}
\newcommand{\Xlen}{\mathit{len}}
\newcommand{\Xundef}{\mathsf{undef}}
\newcommand{\Xstart}{\mathsf{start}}

\newcommand{\given}{\ |\ }
\newcommand{\parm}{\mathord{\color{black!50}\bullet}}

% ========================

%%% TikZ setup

\usepackage{tikz}
\usetikzlibrary{positioning,graphs,automata}
\usetikzlibrary{shapes.multipart}
% \usetikzlibrary{external}
% \tikzexternalize

% AAAI wants at least 0.5pt lines
\tikzset{every picture/.style={line width=0.5pt}}
\tikzset{>=stealth}

\def\XcrossedEdgeFromParent#1#2{
  [style=edge from parent,#1]
  (\tikzparentnode\tikzparentanchor) to #2 node [sloped] {\textsf{x}} (\tikzchildnode\tikzchildanchor)
}

\tikzset{and/.style={draw,circle,fill=black,minimum size=6pt,inner sep=0pt,solid}}
\tikzset{or/.style={draw,circle,fill=white,minimum size=8pt,inner sep=0pt,solid}}
\tikzset{failed node/.style={label=below:{\Lightning}}}
\tikzset{failed child/.style={edge from parent macro=\XcrossedEdgeFromParent}}
\tikzset{unexplored/.style={dashed}}
\tikzset{active/.style={red,very thick}}
\tikzset{myautomata/.style={shorten >=1pt}} %node distance=30mm
\tikzset{initial text={}}  % for initial state of automata
\tikzset{mysplit/.style={rectangle split,rectangle split parts=#1,rectangle split horizontal,draw,rectangle split part align=base,anchor=base,draw=gray}}

%%  one-shot commands for captions etc.

\newcommand{\asdfghjkasdasdasda}{\tikz{\node[and]{};}}
\newcommand{\asdfghjkasdasdasdb}{\tikz{\node[or]{};}}
\newcommand{\asdfghjkasdasdasdc}{\raisebox{-0.4ex}[0pt][0pt]{\tikz{\node[or,double distance=1.5pt] {};}}}
\newcommand{\asdfghjkasdasdasdd}{\tikz[baseline] \node[mysplit=3] {A\nodepart{two}--\nodepart{three}B};}
\newcommand{\asdfghjkasdasdasde}{\tikz[baseline] \node[mysplit=6] {A\nodepart{two}--\nodepart{three}--\nodepart{four}--\nodepart{five}--\nodepart{six}B};}

%% Skulls for the BridgeWalk environment

\usepackage{skull}

% ========================

\newcommand{\interfuncstrut}{\rule{0pt}{3ex}}
